% Generated by Sphinx.
\def\sphinxdocclass{report}
\documentclass[letterpaper,10pt,english]{sphinxmanual}
\usepackage[utf8]{inputenc}
\DeclareUnicodeCharacter{00A0}{\nobreakspace}
\usepackage{cmap}
\usepackage[T1]{fontenc}
\usepackage{babel}
\usepackage{times}
\usepackage[Bjarne]{fncychap}
\usepackage{longtable}
\usepackage{sphinx}
\usepackage{multirow}


\title{Leon Serial Com Documentation}
\date{May 23, 2014}
\release{0.1}
\author{Stavros Giannakopoulos}
\newcommand{\sphinxlogo}{}
\renewcommand{\releasename}{Release}
\makeindex

\makeatletter
\def\PYG@reset{\let\PYG@it=\relax \let\PYG@bf=\relax%
    \let\PYG@ul=\relax \let\PYG@tc=\relax%
    \let\PYG@bc=\relax \let\PYG@ff=\relax}
\def\PYG@tok#1{\csname PYG@tok@#1\endcsname}
\def\PYG@toks#1+{\ifx\relax#1\empty\else%
    \PYG@tok{#1}\expandafter\PYG@toks\fi}
\def\PYG@do#1{\PYG@bc{\PYG@tc{\PYG@ul{%
    \PYG@it{\PYG@bf{\PYG@ff{#1}}}}}}}
\def\PYG#1#2{\PYG@reset\PYG@toks#1+\relax+\PYG@do{#2}}

\expandafter\def\csname PYG@tok@gd\endcsname{\def\PYG@tc##1{\textcolor[rgb]{0.63,0.00,0.00}{##1}}}
\expandafter\def\csname PYG@tok@gu\endcsname{\let\PYG@bf=\textbf\def\PYG@tc##1{\textcolor[rgb]{0.50,0.00,0.50}{##1}}}
\expandafter\def\csname PYG@tok@gt\endcsname{\def\PYG@tc##1{\textcolor[rgb]{0.00,0.27,0.87}{##1}}}
\expandafter\def\csname PYG@tok@gs\endcsname{\let\PYG@bf=\textbf}
\expandafter\def\csname PYG@tok@gr\endcsname{\def\PYG@tc##1{\textcolor[rgb]{1.00,0.00,0.00}{##1}}}
\expandafter\def\csname PYG@tok@cm\endcsname{\let\PYG@it=\textit\def\PYG@tc##1{\textcolor[rgb]{0.25,0.50,0.56}{##1}}}
\expandafter\def\csname PYG@tok@vg\endcsname{\def\PYG@tc##1{\textcolor[rgb]{0.73,0.38,0.84}{##1}}}
\expandafter\def\csname PYG@tok@m\endcsname{\def\PYG@tc##1{\textcolor[rgb]{0.13,0.50,0.31}{##1}}}
\expandafter\def\csname PYG@tok@mh\endcsname{\def\PYG@tc##1{\textcolor[rgb]{0.13,0.50,0.31}{##1}}}
\expandafter\def\csname PYG@tok@cs\endcsname{\def\PYG@tc##1{\textcolor[rgb]{0.25,0.50,0.56}{##1}}\def\PYG@bc##1{\setlength{\fboxsep}{0pt}\colorbox[rgb]{1.00,0.94,0.94}{\strut ##1}}}
\expandafter\def\csname PYG@tok@ge\endcsname{\let\PYG@it=\textit}
\expandafter\def\csname PYG@tok@vc\endcsname{\def\PYG@tc##1{\textcolor[rgb]{0.73,0.38,0.84}{##1}}}
\expandafter\def\csname PYG@tok@il\endcsname{\def\PYG@tc##1{\textcolor[rgb]{0.13,0.50,0.31}{##1}}}
\expandafter\def\csname PYG@tok@go\endcsname{\def\PYG@tc##1{\textcolor[rgb]{0.20,0.20,0.20}{##1}}}
\expandafter\def\csname PYG@tok@cp\endcsname{\def\PYG@tc##1{\textcolor[rgb]{0.00,0.44,0.13}{##1}}}
\expandafter\def\csname PYG@tok@gi\endcsname{\def\PYG@tc##1{\textcolor[rgb]{0.00,0.63,0.00}{##1}}}
\expandafter\def\csname PYG@tok@gh\endcsname{\let\PYG@bf=\textbf\def\PYG@tc##1{\textcolor[rgb]{0.00,0.00,0.50}{##1}}}
\expandafter\def\csname PYG@tok@ni\endcsname{\let\PYG@bf=\textbf\def\PYG@tc##1{\textcolor[rgb]{0.84,0.33,0.22}{##1}}}
\expandafter\def\csname PYG@tok@nl\endcsname{\let\PYG@bf=\textbf\def\PYG@tc##1{\textcolor[rgb]{0.00,0.13,0.44}{##1}}}
\expandafter\def\csname PYG@tok@nn\endcsname{\let\PYG@bf=\textbf\def\PYG@tc##1{\textcolor[rgb]{0.05,0.52,0.71}{##1}}}
\expandafter\def\csname PYG@tok@no\endcsname{\def\PYG@tc##1{\textcolor[rgb]{0.38,0.68,0.84}{##1}}}
\expandafter\def\csname PYG@tok@na\endcsname{\def\PYG@tc##1{\textcolor[rgb]{0.25,0.44,0.63}{##1}}}
\expandafter\def\csname PYG@tok@nb\endcsname{\def\PYG@tc##1{\textcolor[rgb]{0.00,0.44,0.13}{##1}}}
\expandafter\def\csname PYG@tok@nc\endcsname{\let\PYG@bf=\textbf\def\PYG@tc##1{\textcolor[rgb]{0.05,0.52,0.71}{##1}}}
\expandafter\def\csname PYG@tok@nd\endcsname{\let\PYG@bf=\textbf\def\PYG@tc##1{\textcolor[rgb]{0.33,0.33,0.33}{##1}}}
\expandafter\def\csname PYG@tok@ne\endcsname{\def\PYG@tc##1{\textcolor[rgb]{0.00,0.44,0.13}{##1}}}
\expandafter\def\csname PYG@tok@nf\endcsname{\def\PYG@tc##1{\textcolor[rgb]{0.02,0.16,0.49}{##1}}}
\expandafter\def\csname PYG@tok@si\endcsname{\let\PYG@it=\textit\def\PYG@tc##1{\textcolor[rgb]{0.44,0.63,0.82}{##1}}}
\expandafter\def\csname PYG@tok@s2\endcsname{\def\PYG@tc##1{\textcolor[rgb]{0.25,0.44,0.63}{##1}}}
\expandafter\def\csname PYG@tok@vi\endcsname{\def\PYG@tc##1{\textcolor[rgb]{0.73,0.38,0.84}{##1}}}
\expandafter\def\csname PYG@tok@nt\endcsname{\let\PYG@bf=\textbf\def\PYG@tc##1{\textcolor[rgb]{0.02,0.16,0.45}{##1}}}
\expandafter\def\csname PYG@tok@nv\endcsname{\def\PYG@tc##1{\textcolor[rgb]{0.73,0.38,0.84}{##1}}}
\expandafter\def\csname PYG@tok@s1\endcsname{\def\PYG@tc##1{\textcolor[rgb]{0.25,0.44,0.63}{##1}}}
\expandafter\def\csname PYG@tok@gp\endcsname{\let\PYG@bf=\textbf\def\PYG@tc##1{\textcolor[rgb]{0.78,0.36,0.04}{##1}}}
\expandafter\def\csname PYG@tok@sh\endcsname{\def\PYG@tc##1{\textcolor[rgb]{0.25,0.44,0.63}{##1}}}
\expandafter\def\csname PYG@tok@ow\endcsname{\let\PYG@bf=\textbf\def\PYG@tc##1{\textcolor[rgb]{0.00,0.44,0.13}{##1}}}
\expandafter\def\csname PYG@tok@sx\endcsname{\def\PYG@tc##1{\textcolor[rgb]{0.78,0.36,0.04}{##1}}}
\expandafter\def\csname PYG@tok@bp\endcsname{\def\PYG@tc##1{\textcolor[rgb]{0.00,0.44,0.13}{##1}}}
\expandafter\def\csname PYG@tok@c1\endcsname{\let\PYG@it=\textit\def\PYG@tc##1{\textcolor[rgb]{0.25,0.50,0.56}{##1}}}
\expandafter\def\csname PYG@tok@kc\endcsname{\let\PYG@bf=\textbf\def\PYG@tc##1{\textcolor[rgb]{0.00,0.44,0.13}{##1}}}
\expandafter\def\csname PYG@tok@c\endcsname{\let\PYG@it=\textit\def\PYG@tc##1{\textcolor[rgb]{0.25,0.50,0.56}{##1}}}
\expandafter\def\csname PYG@tok@mf\endcsname{\def\PYG@tc##1{\textcolor[rgb]{0.13,0.50,0.31}{##1}}}
\expandafter\def\csname PYG@tok@err\endcsname{\def\PYG@bc##1{\setlength{\fboxsep}{0pt}\fcolorbox[rgb]{1.00,0.00,0.00}{1,1,1}{\strut ##1}}}
\expandafter\def\csname PYG@tok@kd\endcsname{\let\PYG@bf=\textbf\def\PYG@tc##1{\textcolor[rgb]{0.00,0.44,0.13}{##1}}}
\expandafter\def\csname PYG@tok@ss\endcsname{\def\PYG@tc##1{\textcolor[rgb]{0.32,0.47,0.09}{##1}}}
\expandafter\def\csname PYG@tok@sr\endcsname{\def\PYG@tc##1{\textcolor[rgb]{0.14,0.33,0.53}{##1}}}
\expandafter\def\csname PYG@tok@mo\endcsname{\def\PYG@tc##1{\textcolor[rgb]{0.13,0.50,0.31}{##1}}}
\expandafter\def\csname PYG@tok@mi\endcsname{\def\PYG@tc##1{\textcolor[rgb]{0.13,0.50,0.31}{##1}}}
\expandafter\def\csname PYG@tok@kn\endcsname{\let\PYG@bf=\textbf\def\PYG@tc##1{\textcolor[rgb]{0.00,0.44,0.13}{##1}}}
\expandafter\def\csname PYG@tok@o\endcsname{\def\PYG@tc##1{\textcolor[rgb]{0.40,0.40,0.40}{##1}}}
\expandafter\def\csname PYG@tok@kr\endcsname{\let\PYG@bf=\textbf\def\PYG@tc##1{\textcolor[rgb]{0.00,0.44,0.13}{##1}}}
\expandafter\def\csname PYG@tok@s\endcsname{\def\PYG@tc##1{\textcolor[rgb]{0.25,0.44,0.63}{##1}}}
\expandafter\def\csname PYG@tok@kp\endcsname{\def\PYG@tc##1{\textcolor[rgb]{0.00,0.44,0.13}{##1}}}
\expandafter\def\csname PYG@tok@w\endcsname{\def\PYG@tc##1{\textcolor[rgb]{0.73,0.73,0.73}{##1}}}
\expandafter\def\csname PYG@tok@kt\endcsname{\def\PYG@tc##1{\textcolor[rgb]{0.56,0.13,0.00}{##1}}}
\expandafter\def\csname PYG@tok@sc\endcsname{\def\PYG@tc##1{\textcolor[rgb]{0.25,0.44,0.63}{##1}}}
\expandafter\def\csname PYG@tok@sb\endcsname{\def\PYG@tc##1{\textcolor[rgb]{0.25,0.44,0.63}{##1}}}
\expandafter\def\csname PYG@tok@k\endcsname{\let\PYG@bf=\textbf\def\PYG@tc##1{\textcolor[rgb]{0.00,0.44,0.13}{##1}}}
\expandafter\def\csname PYG@tok@se\endcsname{\let\PYG@bf=\textbf\def\PYG@tc##1{\textcolor[rgb]{0.25,0.44,0.63}{##1}}}
\expandafter\def\csname PYG@tok@sd\endcsname{\let\PYG@it=\textit\def\PYG@tc##1{\textcolor[rgb]{0.25,0.44,0.63}{##1}}}

\def\PYGZbs{\char`\\}
\def\PYGZus{\char`\_}
\def\PYGZob{\char`\{}
\def\PYGZcb{\char`\}}
\def\PYGZca{\char`\^}
\def\PYGZam{\char`\&}
\def\PYGZlt{\char`\<}
\def\PYGZgt{\char`\>}
\def\PYGZsh{\char`\#}
\def\PYGZpc{\char`\%}
\def\PYGZdl{\char`\$}
\def\PYGZhy{\char`\-}
\def\PYGZsq{\char`\'}
\def\PYGZdq{\char`\"}
\def\PYGZti{\char`\~}
% for compatibility with earlier versions
\def\PYGZat{@}
\def\PYGZlb{[}
\def\PYGZrb{]}
\makeatother

\begin{document}

\maketitle
\tableofcontents
\phantomsection\label{index::doc}


Contents:


\chapter{Introduction}
\label{Introduction:introduction}\label{Introduction::doc}\label{Introduction:welcome-to-leon-serial-com-s-documentation}
This is the introduction of the Leon-Serial documentation.


\chapter{Auto Generated Documentation}
\label{Code:auto-generated-documentation}\label{Code::doc}

\section{Serial Communication}
\label{Code:module-leonSer}\label{Code:serial-communication}\index{leonSer (module)}
The leonSer module includes functions for handling the communication through serial port on the 
most basic level.

The current setup automatically detects the OS and opens the port accordingly (see \code{platform}). 
It includes 4 methods for opening, communicating and closing serial ports.

Author: Stavros Giannakopoulos
\index{leonTx() (in module leonSer)}

\begin{fulllineitems}
\phantomsection\label{Code:leonSer.leonTx}\pysiglinewithargsret{\code{leonSer.}\bfcode{leonTx}}{\emph{serport}, \emph{strinp}}{}
This is the second function which is used to send data to the Communication Port.
It takes a port object, in which it sends the string object also received after concatenating 
the `endline' character to the end.     
It finally receives the answer from the board  up to 100 bytes and it prints it to the terminal.
It may be modified to return the answer as a string.
\begin{quote}\begin{description}
\item[{Parameters}] \leavevmode\begin{itemize}
\item {} 
\textbf{serport} -- Receives an open serial port object as an input.

\item {} 
\textbf{strinp} -- Receives a string to be sent to the com- port.

\end{itemize}

\end{description}\end{quote}

\end{fulllineitems}

\index{leonsend() (in module leonSer)}

\begin{fulllineitems}
\phantomsection\label{Code:leonSer.leonsend}\pysiglinewithargsret{\code{leonSer.}\bfcode{leonsend}}{\emph{serport}, \emph{strinp}}{}
This is the second function which is used to send and receive data to the Communication Port.
It takes a port object, in which it sends the string object also received after concatenating 
the `endline' character to the end.     
It finally receives the answer from the board  up to 100 bytes and it prints it to the terminal.
It may be modified to return the answer as a string.
\begin{quote}\begin{description}
\item[{Parameters}] \leavevmode\begin{itemize}
\item {} 
\textbf{serport} -- Receives an open serial port object as an input.

\item {} 
\textbf{strinp} -- Receives a string to be sent to the com- port.

\end{itemize}

\item[{Returns}] \leavevmode
s, which is a string read by the serial port.

\end{description}\end{quote}

\end{fulllineitems}

\index{leonstart() (in module leonSer)}

\begin{fulllineitems}
\phantomsection\label{Code:leonSer.leonstart}\pysiglinewithargsret{\code{leonSer.}\bfcode{leonstart}}{}{}
This is the first function which is used to initialize the Communication Port. 
Returns the opened port. The Com5 port is set to open for now, but it might be 
implemented as an argument- or detected automatically in a later date.
The baudrate is set to 38343 and it might be implemented as an argument in a later build.
\begin{quote}\begin{description}
\item[{Returns}] \leavevmode
serport, which is the serial port object that was opened.

\end{description}\end{quote}

\end{fulllineitems}

\index{leonstop() (in module leonSer)}

\begin{fulllineitems}
\phantomsection\label{Code:leonSer.leonstop}\pysiglinewithargsret{\code{leonSer.}\bfcode{leonstop}}{\emph{serport}}{}
This is the third function which is used to close the Communication Port supplemented as an 
argument while printing a debugging message on the terminal.
\begin{quote}\begin{description}
\item[{Parameters}] \leavevmode
\textbf{serport} -- Receives an open serial port object as an input.

\end{description}\end{quote}

\end{fulllineitems}



\bigskip\hrule{}\bigskip

\phantomsection\label{Code:module-SeriL}\index{SeriL (module)}
The SeriL module includes functions for calling and managing the communication through use of 
the `leonSer' module.

The current setup includes two functions, one for running the communication from the GUI, and one for
user triggered communication for debugging purposes.
\begin{quote}

Author: Stavros Giannakopoulos
\end{quote}
\index{GuiLeon() (in module SeriL)}

\begin{fulllineitems}
\phantomsection\label{Code:SeriL.GuiLeon}\pysiglinewithargsret{\code{SeriL.}\bfcode{GuiLeon}}{\emph{inp}}{}
This function takes a input string and then calls `leonstart' to open a serial port. 
Then it sends the input argument over with `leonsend' on the port opened.
\begin{quote}\begin{description}
\item[{Parameters}] \leavevmode
\textbf{inp} -- Input string to be sent over the serial port.

\end{description}\end{quote}

\end{fulllineitems}

\index{SeriLeon() (in module SeriL)}

\begin{fulllineitems}
\phantomsection\label{Code:SeriL.SeriLeon}\pysiglinewithargsret{\code{SeriL.}\bfcode{SeriLeon}}{\emph{inp}}{}
This function takes a input string and then calls `leonstart' to open a serial port. 
Then it sends the input argument over with `leonsend' on the port opened.
Finally it captures the response of the device.
\begin{quote}\begin{description}
\item[{Parameters}] \leavevmode
\textbf{inp} -- Input string to be sent over the serial port.

\item[{Returns}] \leavevmode
y, a string containing the response of the Leon3 board.

\end{description}\end{quote}

\end{fulllineitems}



\bigskip\hrule{}\bigskip

\phantomsection\label{Code:module-ahbSeri}\index{ahbSeri (module)}
The ahbSeri module is used to manage and debug the :mod:'SeriL' module for debugging reasons
by simulating user triggered read and write functions on the Leon3 board AHB uart.
\begin{quote}

Author: Stavros Giannakopoulos
\end{quote}
\index{ahbread() (in module ahbSeri)}

\begin{fulllineitems}
\phantomsection\label{Code:ahbSeri.ahbread}\pysiglinewithargsret{\code{ahbSeri.}\bfcode{ahbread}}{\emph{addrr}}{}
This function is used to receive a 32-bit hex memory address in the form of a string by the user. 
And then it reads a 32 bit hex number from this address and prints it to the terminal.
In a future build it will return the value of the memory.
\begin{quote}\begin{description}
\item[{Parameters}] \leavevmode
\textbf{addrr} -- Input string , address in the form `0x\#\#\#\#\#\#\#\#'

\item[{Returns}] \leavevmode
The data read by the memory address given.

\end{description}\end{quote}

\end{fulllineitems}

\index{ahbwrite() (in module ahbSeri)}

\begin{fulllineitems}
\phantomsection\label{Code:ahbSeri.ahbwrite}\pysiglinewithargsret{\code{ahbSeri.}\bfcode{ahbwrite}}{\emph{addrr}, \emph{data}}{}
This function is used to write a 32-bit hex number in a 32-bit memory address in the form of a 
string given to the function as an argument.
\begin{quote}\begin{description}
\item[{Parameters}] \leavevmode\begin{itemize}
\item {} 
\textbf{addrr} -- Input string , address in the form `0x\#\#\#\#\#\#\#\#'

\item {} 
\textbf{data} -- Input string, data to be written on the memory, in the form `0x\#\#\#\#\#\#\#\#'

\end{itemize}

\item[{Returns}] \leavevmode
0 if the write was successful or -1 if there was a problem in the communication.

\end{description}\end{quote}

\end{fulllineitems}

\index{userinput() (in module ahbSeri)}

\begin{fulllineitems}
\phantomsection\label{Code:ahbSeri.userinput}\pysiglinewithargsret{\code{ahbSeri.}\bfcode{userinput}}{}{}
This function is used as the main function of this module if the function is called by the terminal.
It simulates the input by the data handling function. 
The user inputs the mode:
\begin{quote}

R= Read
W= Write
\end{quote}

Depending on the mode, the respective function is called and executed with the user input as its arguments.

\end{fulllineitems}



\section{Data handling}
\label{Code:data-handling}\label{Code:module-DataStrLeon}\index{DataStrLeon (module)}
The DataStrLeon module is used to parse and handle the input data by the GUI to the Serial Communication.

Author: Stavros Giannakopoulos
\index{addresser() (in module DataStrLeon)}

\begin{fulllineitems}
\phantomsection\label{Code:DataStrLeon.addresser}\pysiglinewithargsret{\code{DataStrLeon.}\bfcode{addresser}}{\emph{addressinstring}, \emph{words}, \emph{wordlen}}{}
This function takes a startng address and a number of words, and generates a list of addresses sparated by the word
length.
\begin{quote}\begin{description}
\item[{Parameters}] \leavevmode\begin{itemize}
\item {} 
\textbf{addressinstring} -- Is the starting hex address in string format.

\item {} 
\textbf{words} -- Is the number of words addresses that will be generated beyond the first.

\item {} 
\textbf{wordlen} -- Is the length of the word measured in bytes.

\end{itemize}

\item[{Returns}] \leavevmode
addresslist a list of Hex addresses.

\end{description}\end{quote}

\end{fulllineitems}

\index{data4intformator() (in module DataStrLeon)}

\begin{fulllineitems}
\phantomsection\label{Code:DataStrLeon.data4intformator}\pysiglinewithargsret{\code{DataStrLeon.}\bfcode{data4intformator}}{\emph{FloatInt\_list}, \emph{Multipliers}, \emph{Adders}}{}
This function receives a list of float and int numbers and applies the proper multipliers and
additions in order to eliminate any decimals or negative numbers.
\begin{quote}\begin{description}
\item[{Parameters}] \leavevmode\begin{itemize}
\item {} 
\textbf{FloatInt\_list} -- A list float and int numbers.

\item {} 
\textbf{Multipliers} -- A list with multiplier values for the various components.

\item {} 
\textbf{Adders} -- A list with adder values for the EQ calculation.

\end{itemize}

\item[{Returns}] \leavevmode
Grouped\_data, a list of float and int data properly formated.

\end{description}\end{quote}

\end{fulllineitems}

\index{guiparse() (in module DataStrLeon)}

\begin{fulllineitems}
\phantomsection\label{Code:DataStrLeon.guiparse}\pysiglinewithargsret{\code{DataStrLeon.}\bfcode{guiparse}}{\emph{guinput}}{}
This function receives a string from the GUI. Each component is separated by `\#' and each parameter
of the respective component is separated by `,'.
It separates the string into a list of strings with each cell being a coefficient.
\begin{quote}\begin{description}
\item[{Parameters}] \leavevmode
\textbf{guinput} -- String that has proper formating.

\item[{Returns}] \leavevmode
datalist, a list of strings. The dimensions are defined by the input string separators.

\end{description}\end{quote}

\end{fulllineitems}

\index{hexconcatenator() (in module DataStrLeon)}

\begin{fulllineitems}
\phantomsection\label{Code:DataStrLeon.hexconcatenator}\pysiglinewithargsret{\code{DataStrLeon.}\bfcode{hexconcatenator}}{\emph{Inhexed\_list}, \emph{In\_priority}}{}
This function receives a list of strings representing hex numbers and 
concatenates them into 32 bit hex numbers for sending over the serial medium.
\begin{quote}\begin{description}
\item[{Parameters}] \leavevmode\begin{itemize}
\item {} 
\textbf{Inhexed\_list} -- A list of strings representing hex numbers.

\item {} 
\textbf{In\_priority} -- A string of a hex number with the priority of the effects already encoded.

\end{itemize}

\item[{Returns}] \leavevmode
packets, a list of strings ready to send over the \emph{ahbSeri} module.

\end{description}\end{quote}

\end{fulllineitems}

\index{hexizer() (in module DataStrLeon)}

\begin{fulllineitems}
\phantomsection\label{Code:DataStrLeon.hexizer}\pysiglinewithargsret{\code{DataStrLeon.}\bfcode{hexizer}}{\emph{numbered\_data}}{}
This function converts a mixed list of float and int numbers into a list of hex numbers.
\begin{quote}\begin{description}
\item[{Parameters}] \leavevmode
\textbf{numbered\_data} -- List of Float and Int numbers.

\item[{Returns}] \leavevmode
final\_data, a list of string hex numbers. The dimensions are defined by the \emph{lengthshouldbe} list.

\end{description}\end{quote}

\end{fulllineitems}

\index{kickoff() (in module DataStrLeon)}

\begin{fulllineitems}
\phantomsection\label{Code:DataStrLeon.kickoff}\pysiglinewithargsret{\code{DataStrLeon.}\bfcode{kickoff}}{\emph{StrGuiinput}}{}
This function starts the execution of the module. It handles and sends the values to the proper 
functions. 
In this build it is also used for debugging by forcing input and printing the output of the module and 
by sending the data to the \emph{ahbSeri} module.f

\end{fulllineitems}

\index{numerizer() (in module DataStrLeon)}

\begin{fulllineitems}
\phantomsection\label{Code:DataStrLeon.numerizer}\pysiglinewithargsret{\code{DataStrLeon.}\bfcode{numerizer}}{\emph{instring}}{}
This function receives a number in the form of a string and it converts it in its 
proper numeric form respectively.
\begin{quote}\begin{description}
\item[{Parameters}] \leavevmode
\textbf{instring} -- A string representing a float or int number.

\item[{Returns}] \leavevmode
the float or int equivalent of the input string respectively.

\end{description}\end{quote}

\end{fulllineitems}

\index{parsed2values() (in module DataStrLeon)}

\begin{fulllineitems}
\phantomsection\label{Code:DataStrLeon.parsed2values}\pysiglinewithargsret{\code{DataStrLeon.}\bfcode{parsed2values}}{\emph{parsed\_data}}{}
This function receives a list of strings from the \emph{guiparse} function. 
It parses the list of strings supplied and then outputs the int or float equivalents in a simlar list.
The numbers are parsed through the \emph{numerizer} function due to the distinction between string to float and
string to int conversions, to avoid exceptions.
\begin{quote}\begin{description}
\item[{Parameters}] \leavevmode
\textbf{parsed\_data} -- A list of strings.

\item[{Returns}] \leavevmode
Outbound\_data, a list of int and float numbers. The dimensions are equal to the input list.

\end{description}\end{quote}

\end{fulllineitems}

\index{prioritizer() (in module DataStrLeon)}

\begin{fulllineitems}
\phantomsection\label{Code:DataStrLeon.prioritizer}\pysiglinewithargsret{\code{DataStrLeon.}\bfcode{prioritizer}}{\emph{In\_priority\_list}, \emph{EffectNum}}{}
This function receives a list of numbers that indicate the priority of each effect. The most significant decimal digit 
indicates the first effect, and so on up to 9 effects. A 0 for a priority means that the effect is disabled.
The order of the effect is as follows: 
Delay, Chorus, Flanger,  Tremolo, Vibrato, Modulating Wah wah, Auto Wah wah, Phaser and Distortion.
\begin{quote}\begin{description}
\item[{Parameters}] \leavevmode\begin{itemize}
\item {} 
\textbf{In\_priority\_list} -- a list of string decimal numbers from 0-9. Each number corresponds to one effect.

\item {} 
\textbf{EffectNum} -- Is the number of effects passed down by the kickoff function.

\end{itemize}

\item[{Returns}] \leavevmode
hexoutput, a string converted hex number that includes the effects encoded.

\end{description}\end{quote}

\end{fulllineitems}



\section{Graphic User Interface}
\label{Code:graphic-user-interface}\label{Code:module-GUI}\index{GUI (module)}
The GUI module is used create and handle the Graphic User Interface for setting the
effect coefficients of the Soundbox.

Author: Mohammed Elghoz
\index{AUTO\_val() (in module GUI)}

\begin{fulllineitems}
\phantomsection\label{Code:GUI.AUTO_val}\pysiglinewithargsret{\code{GUI.}\bfcode{AUTO\_val}}{}{}
Makes the Auto type selectable for the Wah Wah and displays it if selected. Calls the outp function and saves the selected 
type on the string.

\end{fulllineitems}

\index{BLUES\_val() (in module GUI)}

\begin{fulllineitems}
\phantomsection\label{Code:GUI.BLUES_val}\pysiglinewithargsret{\code{GUI.}\bfcode{BLUES\_val}}{}{}
Makes the Blues type selectable for the Distortion and displays it if selected. Calls the outp function and saves the selected 
type on the string.

\end{fulllineitems}

\index{Chorus() (in module GUI)}

\begin{fulllineitems}
\phantomsection\label{Code:GUI.Chorus}\pysiglinewithargsret{\code{GUI.}\bfcode{Chorus}}{}{}
This function makes sure that the only frame displayed is the Chorus frame. The sliders and Type menus placement 
in the frame is set.

\end{fulllineitems}

\index{Create\_labels() (in module GUI)}

\begin{fulllineitems}
\phantomsection\label{Code:GUI.Create_labels}\pysiglinewithargsret{\code{GUI.}\bfcode{Create\_labels}}{}{}
This function creates the labels where the priority order will be stored.

\end{fulllineitems}

\index{Delay() (in module GUI)}

\begin{fulllineitems}
\phantomsection\label{Code:GUI.Delay}\pysiglinewithargsret{\code{GUI.}\bfcode{Delay}}{}{}
This function makes sure that the only frame displayed is the Delay frame. The sliders and the entry slots placement in
the frame is set.

\end{fulllineitems}

\index{Distortion() (in module GUI)}

\begin{fulllineitems}
\phantomsection\label{Code:GUI.Distortion}\pysiglinewithargsret{\code{GUI.}\bfcode{Distortion}}{}{}
This function makes sure that the only frame displayed is the Distortion frame. The sliders and menu types placement 
in the frame is set.

\end{fulllineitems}

\index{EQ() (in module GUI)}

\begin{fulllineitems}
\phantomsection\label{Code:GUI.EQ}\pysiglinewithargsret{\code{GUI.}\bfcode{EQ}}{}{}
This function makes sure that the only frame displayed is the Equalizer frame. The sliders and the entry slots placement in
the frame is set.

\end{fulllineitems}

\index{Flanger() (in module GUI)}

\begin{fulllineitems}
\phantomsection\label{Code:GUI.Flanger}\pysiglinewithargsret{\code{GUI.}\bfcode{Flanger}}{}{}
This function makes sure that the only frame displayed is the Flanger frame. The sliders placement in the frame is set.

\end{fulllineitems}

\index{Gains() (in module GUI)}

\begin{fulllineitems}
\phantomsection\label{Code:GUI.Gains}\pysiglinewithargsret{\code{GUI.}\bfcode{Gains}}{}{}
This function makes sure that the only frame displayed is the Gains frame. The sliders placement in the frame is set.

\end{fulllineitems}

\index{METAL\_val() (in module GUI)}

\begin{fulllineitems}
\phantomsection\label{Code:GUI.METAL_val}\pysiglinewithargsret{\code{GUI.}\bfcode{METAL\_val}}{}{}
Makes the Metal type selectable for the Distortion and displays it if selected. Calls the outp function and saves the selected 
type on the string.

\end{fulllineitems}

\index{MOD\_val() (in module GUI)}

\begin{fulllineitems}
\phantomsection\label{Code:GUI.MOD_val}\pysiglinewithargsret{\code{GUI.}\bfcode{MOD\_val}}{}{}
Makes the Modulating type selectable for the Wah Wah and displays it if selected. Calls the outp function and saves the selected 
type on the string.

\end{fulllineitems}

\index{NoiseGate() (in module GUI)}

\begin{fulllineitems}
\phantomsection\label{Code:GUI.NoiseGate}\pysiglinewithargsret{\code{GUI.}\bfcode{NoiseGate}}{}{}
This function makes sure that the only frame displayed is the Noisegate frame. The sliders placement in the frame is set.

\end{fulllineitems}

\index{Phaser() (in module GUI)}

\begin{fulllineitems}
\phantomsection\label{Code:GUI.Phaser}\pysiglinewithargsret{\code{GUI.}\bfcode{Phaser}}{}{}
This function makes sure that the only frame displayed is the Phaser frame. The sliders placement 
in the frame is set.

\end{fulllineitems}

\index{ROCK\_val() (in module GUI)}

\begin{fulllineitems}
\phantomsection\label{Code:GUI.ROCK_val}\pysiglinewithargsret{\code{GUI.}\bfcode{ROCK\_val}}{}{}
Makes the Rock type selectable for the Distortion and displays it if selected. Calls the outp function and saves the selected 
type on the string.

\end{fulllineitems}

\index{SetPriority() (in module GUI)}

\begin{fulllineitems}
\phantomsection\label{Code:GUI.SetPriority}\pysiglinewithargsret{\code{GUI.}\bfcode{SetPriority}}{}{}
This function handles the creation of the 12 priority checkboxes.
the effects are as follows: 1:'delay',2:'chorus',3:'flanger',4:'tremolo',5:'vibrato',6:'wah wah',7:'phaser',8:'distortion',9:'noise gate',10:'PreGain',11:'OutGain',12:'EQ'

\end{fulllineitems}

\index{Tremolo() (in module GUI)}

\begin{fulllineitems}
\phantomsection\label{Code:GUI.Tremolo}\pysiglinewithargsret{\code{GUI.}\bfcode{Tremolo}}{}{}
This function makes sure that the only frame displayed is the Tremolo frame. The sliders and the menu type placement 
in the frame is set.

\end{fulllineitems}

\index{Vibrato() (in module GUI)}

\begin{fulllineitems}
\phantomsection\label{Code:GUI.Vibrato}\pysiglinewithargsret{\code{GUI.}\bfcode{Vibrato}}{}{}
This function makes sure that the only frame displayed is the Vibrato frame. The sliders and the menu type placement 
in the frame is set.

\end{fulllineitems}

\index{WahWah() (in module GUI)}

\begin{fulllineitems}
\phantomsection\label{Code:GUI.WahWah}\pysiglinewithargsret{\code{GUI.}\bfcode{WahWah}}{}{}
This function makes sure that the only frame displayed is the Wah Wah frame. The sliders and the menu type placement 
in the frame is set.

\end{fulllineitems}

\index{checkClicked() (in module GUI)}

\begin{fulllineitems}
\phantomsection\label{Code:GUI.checkClicked}\pysiglinewithargsret{\code{GUI.}\bfcode{checkClicked}}{\emph{number}}{}
This function takes is called whenever a priority item is clicked. It takes the number of the priority item that called it as 
an input and places the appropriate effect on the priority list frame.
:param name: Number that coresponds to a priority checkbox clicked.

\end{fulllineitems}

\index{chorusRANDOM\_val() (in module GUI)}

\begin{fulllineitems}
\phantomsection\label{Code:GUI.chorusRANDOM_val}\pysiglinewithargsret{\code{GUI.}\bfcode{chorusRANDOM\_val}}{}{}
Makes the Random type selectable for the Chorus and displays it if selected. Calls the outp function and saves the selected 
type on the string.

\end{fulllineitems}

\index{chorusSAWTOOTH\_val() (in module GUI)}

\begin{fulllineitems}
\phantomsection\label{Code:GUI.chorusSAWTOOTH_val}\pysiglinewithargsret{\code{GUI.}\bfcode{chorusSAWTOOTH\_val}}{}{}
Makes the Sawtooth type selectable for the Chorus and displays it if selected. Calls the outp function and saves the selected 
type on the string.

\end{fulllineitems}

\index{chorusSINE\_val() (in module GUI)}

\begin{fulllineitems}
\phantomsection\label{Code:GUI.chorusSINE_val}\pysiglinewithargsret{\code{GUI.}\bfcode{chorusSINE\_val}}{}{}
Makes the Sine type selectable and displays it if selected. Calls the outp function and saves the selected 
type on the string.

\end{fulllineitems}

\index{chorusSQUARE\_val() (in module GUI)}

\begin{fulllineitems}
\phantomsection\label{Code:GUI.chorusSQUARE_val}\pysiglinewithargsret{\code{GUI.}\bfcode{chorusSQUARE\_val}}{}{}
Makes the Square type selectable for the Chorus and displays it if selected. Calls the outp function and saves the selected 
type on the string.

\end{fulllineitems}

\index{chorusTRIANGLE\_val() (in module GUI)}

\begin{fulllineitems}
\phantomsection\label{Code:GUI.chorusTRIANGLE_val}\pysiglinewithargsret{\code{GUI.}\bfcode{chorusTRIANGLE\_val}}{}{}
Makes the Triangle type selectable for the Chorus and displays it if selected. Calls the outp function and saves the selected 
type on the string.

\end{fulllineitems}

\index{exitGUI() (in module GUI)}

\begin{fulllineitems}
\phantomsection\label{Code:GUI.exitGUI}\pysiglinewithargsret{\code{GUI.}\bfcode{exitGUI}}{}{}
When the EXIT button is clicked the master frame is closed, thus closing the entire GUI.

\end{fulllineitems}

\index{hello() (in module GUI)}

\begin{fulllineitems}
\phantomsection\label{Code:GUI.hello}\pysiglinewithargsret{\code{GUI.}\bfcode{hello}}{}{}
This function is called from the choosable options from the File, Edit and Help menu that has yet to be configured.
This function just prints the word hello.

\end{fulllineitems}

\index{labellistset() (in module GUI)}

\begin{fulllineitems}
\phantomsection\label{Code:GUI.labellistset}\pysiglinewithargsret{\code{GUI.}\bfcode{labellistset}}{}{}
This function updates the Outputpriorities list that is then sent on the Leon3 via the communication code.

\end{fulllineitems}

\index{main() (in module GUI)}

\begin{fulllineitems}
\phantomsection\label{Code:GUI.main}\pysiglinewithargsret{\code{GUI.}\bfcode{main}}{}{}
Depending on the value of the global variable v, a function of an effect is called and can be modified.

\end{fulllineitems}

\index{makeMenu() (in module GUI)}

\begin{fulllineitems}
\phantomsection\label{Code:GUI.makeMenu}\pysiglinewithargsret{\code{GUI.}\bfcode{makeMenu}}{}{}
Creates a File menu, Edit menu, and Help menu.
The File menu has the ability to save the current output string to a text file and reload it. The Edit and Help menu calls the hello
function and prints hello.

\end{fulllineitems}

\index{openFile() (in module GUI)}

\begin{fulllineitems}
\phantomsection\label{Code:GUI.openFile}\pysiglinewithargsret{\code{GUI.}\bfcode{openFile}}{}{}
Displays the last saved string.

\end{fulllineitems}

\index{outp() (in module GUI)}

\begin{fulllineitems}
\phantomsection\label{Code:GUI.outp}\pysiglinewithargsret{\code{GUI.}\bfcode{outp}}{\emph{strprint}, \emph{position}}{}
This function is called from every function where an effect value can be set and places the chosen values in the correct place.
:param strprint: Places the values in the string.
:param position: Places the values in the chosen position (in the string).

\end{fulllineitems}

\index{radiobuttons() (in module GUI)}

\begin{fulllineitems}
\phantomsection\label{Code:GUI.radiobuttons}\pysiglinewithargsret{\code{GUI.}\bfcode{radiobuttons}}{}{}
Creates a radiobutton for every effect, and assigns the variable v a value depending on which effect is selected.

\end{fulllineitems}

\index{resetbtn() (in module GUI)}

\begin{fulllineitems}
\phantomsection\label{Code:GUI.resetbtn}\pysiglinewithargsret{\code{GUI.}\bfcode{resetbtn}}{}{}
This function resets the effect priorities to their zero values and clears the priority list.

\end{fulllineitems}

\index{saveFile() (in module GUI)}

\begin{fulllineitems}
\phantomsection\label{Code:GUI.saveFile}\pysiglinewithargsret{\code{GUI.}\bfcode{saveFile}}{}{}
Saves the current string to a file.

\end{fulllineitems}

\index{set\_bc() (in module GUI)}

\begin{fulllineitems}
\phantomsection\label{Code:GUI.set_bc}\pysiglinewithargsret{\code{GUI.}\bfcode{set\_bc}}{}{}
This function is called when the PROGRAM button is pressed and calls in turn the output function where it stores the chosen 
value for the Bass cutoff.

\end{fulllineitems}

\index{set\_bq() (in module GUI)}

\begin{fulllineitems}
\phantomsection\label{Code:GUI.set_bq}\pysiglinewithargsret{\code{GUI.}\bfcode{set\_bq}}{}{}
This function is called when the PROGRAM button is pressed and calls in turn the output function where it stores the chosen 
value for the Bass Q.

\end{fulllineitems}

\index{set\_bv() (in module GUI)}

\begin{fulllineitems}
\phantomsection\label{Code:GUI.set_bv}\pysiglinewithargsret{\code{GUI.}\bfcode{set\_bv}}{}{}
This function is called when the PROGRAM button is pressed and calls in turn the output function where it stores the chosen 
value for the Bass gain.

\end{fulllineitems}

\index{set\_chorusDepth() (in module GUI)}

\begin{fulllineitems}
\phantomsection\label{Code:GUI.set_chorusDepth}\pysiglinewithargsret{\code{GUI.}\bfcode{set\_chorusDepth}}{}{}
This function is called when the PROGRAM button is pressed and calls in turn the output function where it stores the chosen 
value for the Chorus depths

\end{fulllineitems}

\index{set\_chorusLevel() (in module GUI)}

\begin{fulllineitems}
\phantomsection\label{Code:GUI.set_chorusLevel}\pysiglinewithargsret{\code{GUI.}\bfcode{set\_chorusLevel}}{}{}
This function is called when the PROGRAM button is pressed and calls in turn the output function where it stores the chosen 
value for the Chorus level.

\end{fulllineitems}

\index{set\_chorusRate() (in module GUI)}

\begin{fulllineitems}
\phantomsection\label{Code:GUI.set_chorusRate}\pysiglinewithargsret{\code{GUI.}\bfcode{set\_chorusRate}}{}{}
This function is called when the PROGRAM button is pressed and calls in turn the output function where it stores the chosen 
value for the Chorus rate.

\end{fulllineitems}

\index{set\_delayDryWet() (in module GUI)}

\begin{fulllineitems}
\phantomsection\label{Code:GUI.set_delayDryWet}\pysiglinewithargsret{\code{GUI.}\bfcode{set\_delayDryWet}}{}{}
This function is called when the PROGRAM button is pressed and calls in turn the output function where it stores the chosen 
value for the Delay Dry/Wet.

\end{fulllineitems}

\index{set\_delayFeedback() (in module GUI)}

\begin{fulllineitems}
\phantomsection\label{Code:GUI.set_delayFeedback}\pysiglinewithargsret{\code{GUI.}\bfcode{set\_delayFeedback}}{}{}
This function is called when the PROGRAM button is pressed and calls in turn the output function where it stores the chosen 
value for the Delay feedback.

\end{fulllineitems}

\index{set\_delayTime() (in module GUI)}

\begin{fulllineitems}
\phantomsection\label{Code:GUI.set_delayTime}\pysiglinewithargsret{\code{GUI.}\bfcode{set\_delayTime}}{}{}
This function is called when the PROGRAM button is pressed and calls in turn the output function where it stores the chosen 
value for the Delay feedback.

\end{fulllineitems}

\index{set\_distortionLevel() (in module GUI)}

\begin{fulllineitems}
\phantomsection\label{Code:GUI.set_distortionLevel}\pysiglinewithargsret{\code{GUI.}\bfcode{set\_distortionLevel}}{}{}
This function is called when the PROGRAM button is pressed and calls in turn the output function where it stores the chosen 
value for the Distortion level.

\end{fulllineitems}

\index{set\_distortionMastergain() (in module GUI)}

\begin{fulllineitems}
\phantomsection\label{Code:GUI.set_distortionMastergain}\pysiglinewithargsret{\code{GUI.}\bfcode{set\_distortionMastergain}}{}{}
This function is called when the PROGRAM button is pressed and calls in turn the output function where it stores the chosen 
value for the Distortion master gain.

\end{fulllineitems}

\index{set\_distortionPregain() (in module GUI)}

\begin{fulllineitems}
\phantomsection\label{Code:GUI.set_distortionPregain}\pysiglinewithargsret{\code{GUI.}\bfcode{set\_distortionPregain}}{}{}
This function is called when the PROGRAM button is pressed and calls in turn the output function where it stores the chosen 
value for the Distortion pregain.

\end{fulllineitems}

\index{set\_distortionTone() (in module GUI)}

\begin{fulllineitems}
\phantomsection\label{Code:GUI.set_distortionTone}\pysiglinewithargsret{\code{GUI.}\bfcode{set\_distortionTone}}{}{}
This function is called when the PROGRAM button is pressed and calls in turn the output function where it stores the chosen 
value for the Distortion tone.

\end{fulllineitems}

\index{set\_flangerDelay() (in module GUI)}

\begin{fulllineitems}
\phantomsection\label{Code:GUI.set_flangerDelay}\pysiglinewithargsret{\code{GUI.}\bfcode{set\_flangerDelay}}{}{}
This function is called when the PROGRAM button is pressed and calls in turn the output function where it stores the chosen 
value for the Flanger delay.

\end{fulllineitems}

\index{set\_flangerDepth() (in module GUI)}

\begin{fulllineitems}
\phantomsection\label{Code:GUI.set_flangerDepth}\pysiglinewithargsret{\code{GUI.}\bfcode{set\_flangerDepth}}{}{}
This function is called when the PROGRAM button is pressed and calls in turn the output function where it stores the chosen 
value for the Flanger depth.

\end{fulllineitems}

\index{set\_flangerLevel() (in module GUI)}

\begin{fulllineitems}
\phantomsection\label{Code:GUI.set_flangerLevel}\pysiglinewithargsret{\code{GUI.}\bfcode{set\_flangerLevel}}{}{}
This function is called when the PROGRAM button is pressed and calls in turn the output function where it stores the chosen 
value for the Flanger level.

\end{fulllineitems}

\index{set\_flangerRate() (in module GUI)}

\begin{fulllineitems}
\phantomsection\label{Code:GUI.set_flangerRate}\pysiglinewithargsret{\code{GUI.}\bfcode{set\_flangerRate}}{}{}
This function is called when the PROGRAM button is pressed and calls in turn the output function where it stores the chosen 
value for the Flanger rate.

\end{fulllineitems}

\index{set\_gain1() (in module GUI)}

\begin{fulllineitems}
\phantomsection\label{Code:GUI.set_gain1}\pysiglinewithargsret{\code{GUI.}\bfcode{set\_gain1}}{}{}
This function is called when the PROGRAM button is pressed and calls in turn the output function where it stores the chosen 
value for the Pregain.

\end{fulllineitems}

\index{set\_gain2() (in module GUI)}

\begin{fulllineitems}
\phantomsection\label{Code:GUI.set_gain2}\pysiglinewithargsret{\code{GUI.}\bfcode{set\_gain2}}{}{}
This function is called when the PROGRAM button is pressed and calls in turn the output function where it stores the chosen 
value for the Outgain.

\end{fulllineitems}

\index{set\_labels() (in module GUI)}

\begin{fulllineitems}
\phantomsection\label{Code:GUI.set_labels}\pysiglinewithargsret{\code{GUI.}\bfcode{set\_labels}}{}{}
Places the variables in the correct order.

\end{fulllineitems}

\index{set\_noisegateThreshold() (in module GUI)}

\begin{fulllineitems}
\phantomsection\label{Code:GUI.set_noisegateThreshold}\pysiglinewithargsret{\code{GUI.}\bfcode{set\_noisegateThreshold}}{}{}
This function is called when the PROGRAM button is pressed and calls in turn the output function where it stores the chosen 
value for the Noisegate threshold.

\end{fulllineitems}

\index{set\_pc() (in module GUI)}

\begin{fulllineitems}
\phantomsection\label{Code:GUI.set_pc}\pysiglinewithargsret{\code{GUI.}\bfcode{set\_pc}}{}{}
This function is called when the PROGRAM button is pressed and calls in turn the output function where it stores the chosen 
value for the Peak cutoff.

\end{fulllineitems}

\index{set\_phaserDepth() (in module GUI)}

\begin{fulllineitems}
\phantomsection\label{Code:GUI.set_phaserDepth}\pysiglinewithargsret{\code{GUI.}\bfcode{set\_phaserDepth}}{}{}
This function is called when the PROGRAM button is pressed and calls in turn the output function where it stores the chosen 
value for the Phaser depth.

\end{fulllineitems}

\index{set\_phaserRate() (in module GUI)}

\begin{fulllineitems}
\phantomsection\label{Code:GUI.set_phaserRate}\pysiglinewithargsret{\code{GUI.}\bfcode{set\_phaserRate}}{}{}
This function is called when the PROGRAM button is pressed and calls in turn the output function where it stores the chosen 
value for the Phaser rate.

\end{fulllineitems}

\index{set\_phaserRes() (in module GUI)}

\begin{fulllineitems}
\phantomsection\label{Code:GUI.set_phaserRes}\pysiglinewithargsret{\code{GUI.}\bfcode{set\_phaserRes}}{}{}
This function is called when the PROGRAM button is pressed and calls in turn the output function where it stores the chosen 
value for the Phaser res.

\end{fulllineitems}

\index{set\_pq() (in module GUI)}

\begin{fulllineitems}
\phantomsection\label{Code:GUI.set_pq}\pysiglinewithargsret{\code{GUI.}\bfcode{set\_pq}}{}{}
This function is called when the PROGRAM button is pressed and calls in turn the output function where it stores the chosen 
value for the Peak Q.

\end{fulllineitems}

\index{set\_pv() (in module GUI)}

\begin{fulllineitems}
\phantomsection\label{Code:GUI.set_pv}\pysiglinewithargsret{\code{GUI.}\bfcode{set\_pv}}{}{}
This function is called when the PROGRAM button is pressed and calls in turn the output function where it stores the chosen 
value for the Peak gain.

\end{fulllineitems}

\index{set\_tc() (in module GUI)}

\begin{fulllineitems}
\phantomsection\label{Code:GUI.set_tc}\pysiglinewithargsret{\code{GUI.}\bfcode{set\_tc}}{}{}
This function is called when the PROGRAM button is pressed and calls in turn the output function where it stores the chosen 
value for the Trebble cutoff.

\end{fulllineitems}

\index{set\_tq() (in module GUI)}

\begin{fulllineitems}
\phantomsection\label{Code:GUI.set_tq}\pysiglinewithargsret{\code{GUI.}\bfcode{set\_tq}}{}{}
This function is called when the PROGRAM button is pressed and calls in turn the output function where it stores the chosen 
value for the Trebble Q.

\end{fulllineitems}

\index{set\_tremoloDepth() (in module GUI)}

\begin{fulllineitems}
\phantomsection\label{Code:GUI.set_tremoloDepth}\pysiglinewithargsret{\code{GUI.}\bfcode{set\_tremoloDepth}}{}{}
This function is called when the PROGRAM button is pressed and calls in turn the output function where it stores the chosen 
value for the Tremolo depth.

\end{fulllineitems}

\index{set\_tremoloLevel() (in module GUI)}

\begin{fulllineitems}
\phantomsection\label{Code:GUI.set_tremoloLevel}\pysiglinewithargsret{\code{GUI.}\bfcode{set\_tremoloLevel}}{}{}
This function is called when the PROGRAM button is pressed and calls in turn the output function where it stores the chosen 
value for the Tremolo level.

\end{fulllineitems}

\index{set\_tremoloRate() (in module GUI)}

\begin{fulllineitems}
\phantomsection\label{Code:GUI.set_tremoloRate}\pysiglinewithargsret{\code{GUI.}\bfcode{set\_tremoloRate}}{}{}
This function is called when the PROGRAM button is pressed and calls in turn the output function where it stores the chosen 
value for the Tremolo rate.

\end{fulllineitems}

\index{set\_tv() (in module GUI)}

\begin{fulllineitems}
\phantomsection\label{Code:GUI.set_tv}\pysiglinewithargsret{\code{GUI.}\bfcode{set\_tv}}{}{}
This function is called when the PROGRAM button is pressed and calls in turn the output function where it stores the chosen 
value for the Trebble gain.

\end{fulllineitems}

\index{set\_vibratoDepth() (in module GUI)}

\begin{fulllineitems}
\phantomsection\label{Code:GUI.set_vibratoDepth}\pysiglinewithargsret{\code{GUI.}\bfcode{set\_vibratoDepth}}{}{}
This function is called when the PROGRAM button is pressed and calls in turn the output function where it stores the chosen 
value for the Vibrato depth.

\end{fulllineitems}

\index{set\_vibratoRate() (in module GUI)}

\begin{fulllineitems}
\phantomsection\label{Code:GUI.set_vibratoRate}\pysiglinewithargsret{\code{GUI.}\bfcode{set\_vibratoRate}}{}{}
This function is called when the PROGRAM button is pressed and calls in turn the output function where it stores the chosen 
value for the Vibrato rate.

\end{fulllineitems}

\index{set\_wahwahDepth() (in module GUI)}

\begin{fulllineitems}
\phantomsection\label{Code:GUI.set_wahwahDepth}\pysiglinewithargsret{\code{GUI.}\bfcode{set\_wahwahDepth}}{}{}
This function is called when the PROGRAM button is pressed and calls in turn the output function where it stores the chosen 
value for the Wah Wah depth.

\end{fulllineitems}

\index{set\_wahwahRate() (in module GUI)}

\begin{fulllineitems}
\phantomsection\label{Code:GUI.set_wahwahRate}\pysiglinewithargsret{\code{GUI.}\bfcode{set\_wahwahRate}}{}{}
This function is called when the PROGRAM button is pressed and calls in turn the output function where it stores the chosen 
value for the Wah Wah rate.

\end{fulllineitems}

\index{set\_wahwahRes() (in module GUI)}

\begin{fulllineitems}
\phantomsection\label{Code:GUI.set_wahwahRes}\pysiglinewithargsret{\code{GUI.}\bfcode{set\_wahwahRes}}{}{}
This function is called when the PROGRAM button is pressed and calls in turn the output function where it stores the chosen 
value for the Wah Wah res.

\end{fulllineitems}

\index{strout2 (in module GUI)}

\begin{fulllineitems}
\phantomsection\label{Code:GUI.strout2}\pysigline{\code{GUI.}\bfcode{strout2}\strong{ = {[}`0', `0', `0', `\#', `0', `0', `0', `\#', `0', `0', `0', `\#', `0', `0', `0', `\#', `0', `0', `0', `0', `\#', `0', `0', `0', `0', `\#', `0', `0', `0', `0', `\#', `0', `0', `0', `\#', `0', `0', `0', `0', `\#', `0', `0', `0', `\#', `0', `0', `0', `0', `0', `\#', `0', `\#', `0', `\#', `0'{]}}}
This is the strings that are outputted from the GUI. The PriorityList keeps track of the order of the effects. The strout2 string
keeps track of all the values set for the different effects. The `\#' is used to separate the effects from each other.
The order of the effects is the following:

Bass gain, Bass Cutoff frequency, Bass Q and then a `\#'.
Peak gain, Peak Cutoff frequency, Peak Q and then a `\#'.
Trebble gain, Trebble   Cutoff frequency, Trebble Q and then a `\#'.
Delay time, Delay feedback, Delay dry/wet and then a `\#'.
Chorus rate, Chorus depth, Chorus level, Chorus type and then a `\#'.
Flanger rate, Flanger depth, Flanger delay, Flanger level and then a `\#'.
Tremolo Rate, Tremolo depth, Tremolo level, Tremolo type and then a `\#'.
Vibrato rate, Vibrato depth, Vibrato type and then a `\#'.
Wah Wah rate, Wah Wah depth, Wah Wah res, Wah Wah type and then a `\#'.
Phaser rate, Phaser depth, Phaser res and then a `\#'.
Distortion pre gain, Distortion master gain, Distortion tone, Distortion level, Distortion type and then a `\#'.
Noise Gate threshold and then a `\#'.
Gain 1 and then a `\#'.
Gain 2 and then a `\#'.

\end{fulllineitems}

\index{tremoloRANDOM\_val() (in module GUI)}

\begin{fulllineitems}
\phantomsection\label{Code:GUI.tremoloRANDOM_val}\pysiglinewithargsret{\code{GUI.}\bfcode{tremoloRANDOM\_val}}{}{}
Makes the Random type selectable for the Tremolo and displays it if selected. Calls the outp function and saves the selected 
type on the string.

\end{fulllineitems}

\index{tremoloSAWTOOTH\_val() (in module GUI)}

\begin{fulllineitems}
\phantomsection\label{Code:GUI.tremoloSAWTOOTH_val}\pysiglinewithargsret{\code{GUI.}\bfcode{tremoloSAWTOOTH\_val}}{}{}
Makes the Sawtooth type selectable for the Tremolo and displays it if selected. Calls the outp function and saves the selected 
type on the string.

\end{fulllineitems}

\index{tremoloSINE\_val() (in module GUI)}

\begin{fulllineitems}
\phantomsection\label{Code:GUI.tremoloSINE_val}\pysiglinewithargsret{\code{GUI.}\bfcode{tremoloSINE\_val}}{}{}
Makes the Sine type selectable for the Tremolo and displays it if selected. Calls the outp function and saves the selected 
type on the string.

\end{fulllineitems}

\index{tremoloSQUARE\_val() (in module GUI)}

\begin{fulllineitems}
\phantomsection\label{Code:GUI.tremoloSQUARE_val}\pysiglinewithargsret{\code{GUI.}\bfcode{tremoloSQUARE\_val}}{}{}
Makes the Square type selectable for the Tremolo and displays it if selected. Calls the outp function and saves the selected 
type on the string.

\end{fulllineitems}

\index{tremoloTRIANGLE\_val() (in module GUI)}

\begin{fulllineitems}
\phantomsection\label{Code:GUI.tremoloTRIANGLE_val}\pysiglinewithargsret{\code{GUI.}\bfcode{tremoloTRIANGLE\_val}}{}{}
Makes the Triangle type selectable for the Tremolo and displays it if selected. Calls the outp function and saves the selected 
type on the string.

\end{fulllineitems}

\index{update\_send() (in module GUI)}

\begin{fulllineitems}
\phantomsection\label{Code:GUI.update_send}\pysiglinewithargsret{\code{GUI.}\bfcode{update\_send}}{}{}
When the Program button is clicked, this function stores the values for all the sliders in the string and outputs them.

\end{fulllineitems}

\index{vibratoRANDOM\_val() (in module GUI)}

\begin{fulllineitems}
\phantomsection\label{Code:GUI.vibratoRANDOM_val}\pysiglinewithargsret{\code{GUI.}\bfcode{vibratoRANDOM\_val}}{}{}
Makes the Random type selectable for the Vibrato and displays it if selected. Calls the outp function and saves the selected 
type on the string.

\end{fulllineitems}

\index{vibratoSAWTOOTH\_val() (in module GUI)}

\begin{fulllineitems}
\phantomsection\label{Code:GUI.vibratoSAWTOOTH_val}\pysiglinewithargsret{\code{GUI.}\bfcode{vibratoSAWTOOTH\_val}}{}{}
Makes the Sawtooth type selectable for the Vibrato and displays it if selected. Calls the outp function and saves the selected 
type on the string.

\end{fulllineitems}

\index{vibratoSINE\_val() (in module GUI)}

\begin{fulllineitems}
\phantomsection\label{Code:GUI.vibratoSINE_val}\pysiglinewithargsret{\code{GUI.}\bfcode{vibratoSINE\_val}}{}{}
Makes the Sine type selectable for the Vibrato and displays it if selected. Calls the outp function and saves the selected 
type on the string.

\end{fulllineitems}

\index{vibratoSQUARE\_val() (in module GUI)}

\begin{fulllineitems}
\phantomsection\label{Code:GUI.vibratoSQUARE_val}\pysiglinewithargsret{\code{GUI.}\bfcode{vibratoSQUARE\_val}}{}{}
Makes the Square type selectable for the Vibrato and displays it if selected. Calls the outp function and saves the selected 
type on the string.

\end{fulllineitems}

\index{vibratoTRIANGLE\_val() (in module GUI)}

\begin{fulllineitems}
\phantomsection\label{Code:GUI.vibratoTRIANGLE_val}\pysiglinewithargsret{\code{GUI.}\bfcode{vibratoTRIANGLE\_val}}{}{}
Makes the Triangle type selectable for the Vibrato and displays it if selected. Calls the outp function and saves the selected 
type on the string.

\end{fulllineitems}



\chapter{Indices and tables}
\label{index:indices-and-tables}\begin{itemize}
\item {} 
\emph{genindex}

\item {} 
\emph{modindex}

\item {} 
\emph{search}

\end{itemize}


\renewcommand{\indexname}{Python Module Index}
\begin{theindex}
\def\bigletter#1{{\Large\sffamily#1}\nopagebreak\vspace{1mm}}
\bigletter{a}
\item {\texttt{ahbSeri}}, \pageref{Code:module-ahbSeri}
\indexspace
\bigletter{d}
\item {\texttt{DataStrLeon}}, \pageref{Code:module-DataStrLeon}
\indexspace
\bigletter{g}
\item {\texttt{GUI}}, \pageref{Code:module-GUI}
\indexspace
\bigletter{l}
\item {\texttt{leonSer}}, \pageref{Code:module-leonSer}
\indexspace
\bigletter{s}
\item {\texttt{SeriL}}, \pageref{Code:module-SeriL}
\end{theindex}

\renewcommand{\indexname}{Index}
\printindex
\end{document}
