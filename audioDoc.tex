\documentclass[12p]{article}
\usepackage[utf8]{inputenc}
\usepackage{cite}
\usepackage{natbib}
\usepackage{fancyhdr}
\usepackage[ampersand]{easylist}
\usepackage[margin=1.4in]{geometry}
\usepackage{verbatim}
\usepackage{graphicx}
\usepackage{wrapfig}

%HEADERS-FOOTERS FOR TITLE PAGE
%-------------------------------------------------------------
\pagestyle{fancy}
\fancyhead[L]{}
\fancyhead[C]{The SoundBox}
\fancyhead[R]{\today}
\fancyfoot[C]{ \thepage}
\fancyfoot[L]{Embedded System Design Project \newline }
\fancyfoot[R]{Team 1 }

%DOCUMENT BEGIN
%------------------------------------------------------------
\begin{document}


%------------------------------------------------------------
%TITLE PAGE
%------------------------------------------------------------
\title{Project Planning}
\author{Project Team: 1}
\date{Version 0.2}


\maketitle 


\vspace*{\fill}
\begin{table}[h]
\centering
\begin{tabular}{|l|l|l|}
\hline
\textbf{Status} & \textbf{Reviewer} & \textbf{Date} \\ \hline
Reviewed        &                   &               \\ \hline
Approved        &                   &               \\ \hline
\end{tabular}
\end{table}

%------------------------------------------------------------
%HEADERS-FOOTERS FOR REST OF DOCUMENT
%------------------------------------------------------------
\thispagestyle{fancy}
\fancyhead[L]{}
\fancyhead[C]{The SoundBox}
\fancyhead[R]{\today}
\renewcommand{\headrulewidth}{0.4pt}
\renewcommand{\footrulewidth}{0.4pt}
\pagenumbering{roman}
\fancyfoot[L]{Embedded System Design Project \newline }
\fancyfoot[R]{Team 1 }
\pagebreak


\vspace*{\fill}
\centering
\section*{Project identity}
\centering
Project group 1, 2013-2014 Spring semester\\
Chalmers Univercity of Technology

%------------------------------------------------------------
%PROJECT GROUP INFORMATION
%------------------------------------------------------------
\begin{table}[h]
\large
\centering
\begin{tabular}{|c|c|c|c|}
\hline
\textbf{Name}                                                    & \textbf{Role}                                                     & \textbf{Phone} & \textbf{Email}              \\ \hline
\begin{tabular}[c]{@{}c@{}}Mohammed \\ Elghoz\end{tabular}       & \begin{tabular}[c]{@{}c@{}}Administrative,\\ Quality\end{tabular} & 0737062592     & elghoz.m@gmail.com          \\ \hline
\begin{tabular}[c]{@{}c@{}}Stavros\\ Giannakopoulos\end{tabular} & Documentation                                                     & 0767651852     & syannako@gmail.com          \\ \hline
Philip Karlsson                                                  & Sound expert                                                      & 0723868160     & philipkarlsson@me.com       \\ \hline
Daniel Moreau                                                    & PM                                                                & 0707280870     & nd.moreau@gmail.com         \\ \hline
Joel Olofsson                                                    & Coding                                                            & 0702714989     & Joel.olofsson@gmail.com     \\ \hline
Jacob Rosén                                                      & Testing                                                           & 0762502835     & jacobro@student.chalmers.se \\ \hline
\end{tabular}
\end{table}




%------------------------------------------------------------
%TOC
%------------------------------------------------------------
\tableofcontents 
\newpage


%------------------------------------------------------------
%DOCUMENT VERSIONS AND CHANGELOG
%------------------------------------------------------------

\begin{table}[h]
\centering
\begin{tabular}{|l|l|l|l|l|}
\hline
\textbf{Version} & \textbf{Date} & \textbf{Changes} & \textbf{Signed} & \textbf{Reviewed} \\ \hline
0.1              &               & First draft      &                 &                   \\ \hline
0.2                 & 102/14/2014              &   Minor Changes               &                 &                   \\ \hline
                 &               &                  &                 &                   \\ \hline
\end{tabular}
\end{table}
\newpage

%-------------------------------------------------------------------
%START OF CHAPTERS
%-------------------------------------------------------------------

\pagenumbering{arabic}
\section{Introduction}
%Small introduction on what you were tasked to do and what you will discuss.

%%%%%%%%%%%%%%%%%%%%%%%%%%%%%%%%%%%%%%%%%%%%%%%%%%%%%%%%%%%%%%%%%%%%%

\newpage
\section{Problem}
\label{Problem}
% In this chapter you elaborate on the task you had, and the technical problem you were called to solve.The chapter takes a form of a scientific problem which is phrased as a scientific question.

%%%%%%%%%%%%%%%%%%%%%%%%%%%%%%%%%%%%%%%%%%%%%%%%%%%%%%%%%%%%%%%%%%%%%
\section{Theory}
\label{Theory}


\subsection{System overview}
\label{Theory:SystemOverview}
%In this chapter we pose the main system directive and design we are asked to do. As well as all the backing theory.

\subsection{Audio}
\label{Theory:Audio}
%In this chapter we discuss what part of the system will be taken care from the team Audio. Include the Effects design and the Sound processing.

\subsection{Digital signal processing}
The digital signal differ from its analogue counter part in two fundamental ways. Firstly, in digital aplications a signal is represented as a discrete stream of values called samples. The rate, the difference in time, between two samples are called the sampling period $T$. The period can be no smaller than the maximum resolution of time offered by the system wich is determined by the system clock. The reciprocal of the sampling period is called the sampling frequency $f_s$. In accordance with the sampling theorem, the sampling frequency should be chosen to be twice of the signal bandwidth component of the sampled signal\emph{fs = fmax} to avoid aliasing. Secondly, the amplitude of the signal is quantizized to a descrete value with the resolution set by the number of avilable bits \cite{udo}. Take for instance a system using twos-complement representation and a w-bits it is possible to represent values between $-2^w$ to $2^w-1$. Consequentially, a digital signal is discrete in both time and amplitude.

\subsubsection{Filters}
In general, the task of filters are to remove or separate components within an object. In signal processing applications filters are used to atenuate, retain or enhance frequency components within a target signal \cite{udo}. The selection of which components and what change e.g in amplitude and phase is done according to frequency. Filters can be built to target different frequencies and in different ways. A myriad of different filters can be be implemented, each changing the amplitude or phase of selected frequency components. Filters can be classified as shown in the bulletlist below \cite{udo}.

\begin{itemize}
\item Lowpass: Retains frequencies up to a set cut-off frequency $f_c$ and attenuates components above.
\item Highpass: Attenuates frequencies up a set $f_c$ and retains components above.
\item Bandpass: Retains components between the low cut-off $f_{cl}$ and high cut-off frequency $f_{ch}$, attenuates other components.
\item Bandreject: Attenuates frequencies between $f_{cl}$ and $f_{ch}$ and retains the rest.
\item Allpass: Retains the amplitude of all components but changes the phase of the signal.
\end{itemize}

\subparagraph
In general signal processing the filters are selected in such a way that unwanted frequeny components are removed from the signal. What remains after the filtering is thus the components of interest.

\subparagraph
Filters can also be used in audio applications as the audio signal is just like any other signal, it contains components of different frequencies. In contrast to other signals, the frequecy components in an audio signal is normaly divided into three categories, the bass (low frequencies), mids (middle high frequencies) and trembel (high frequencies). By applying filters to the audio signal the audio signal can be given a distincly different sound. If a highpass filter with a $f_c$ of about 800 Hz, the bass is effectively removed, leaving a signal that sounds colder than unfiltered signal. In audio applications however, the ordinary filter properties are not desired. Rather than attenuate or retain frequency components, selected components are amplified or attenuated and remaining components retained. Filters that accomplishes this feat is shelving filters and peak filters where the former exist in two different versions \cite{keiler}.

\begin{itemize}
\item Lowpass shelving filter: Attenuates or amplifies frequency components up to a selected $f_c$.
\item Highpass shelving filter: Attenuates or amplifies frequency components from aselected $f_c$.
\item Peak filter: Attenuates or amplifies frequecy components (How to describe?)
\end{itemize}

\subparagraph
By applying different variations of these filters the audio signal can be shaped and the sound altered in desirable ways. Processing an audio signal with respect to frequency is called equalizing, which is a topic that will be persented in later section.

\subparagraph
In audio appliations, filters can be viewed in a different light. Instead of filtering out different frequencies it could instead filter components according to time \cite{udo}. What you get then is the delay based audio effects such as delay, tremolo and chorus all of which is persented in later sections.

\paragraph{FIR filters}

flera tappar

minimal faspåverkan och används med fördel i applikationer som kräver minimal faspåverkan.

Många tunga beräkningar....


\paragraph{IIR filters}
IIR filters is an acronym for infinite impuls respons. The name comes from the fact that IIR filters include a feedback network. Consequeltially, if an impuls is applied at the input the respons will never end \cite{sven}. Because of this feedback network, higher order IIR filters easily become instabil due to rounding errors in the system. In practice, higer order IIR filters are insted realized by connecting several lower order filters (first or second) together to form a series. The second order links in the seires are called biquad links. The term biquad referes to a general second order filter. IIR filters offers tight cotrol of the magnitude respons at the expense of phase. Thus, IIR filters are used when high requirements are posed on the magnitude and phase matters less \cite{storn}. Like the FIR filters, all the afore mentioned filters can be realized as IIR filters by using appropriate filter coefficients. The general equation for a first and second order IIR filter is shown bellow: \\
\null

ADD IN TIME DOMAIN ASWELL!!! THIS IS WHAT WE USE IN THE CODE! Look ad downloaded source...

\begin{equation}
H[z] = \frac{b_0 + b_1*z^{-1}}{1+a_1*z^{-1}}
\end{equation}

\begin{equation}
H[z] = \frac{b_0 + b_1*z^{-1}+b_2*z^{-2}}{1+a_1*z^{-1}+a_2*z^{-2}}
\end{equation}

This form is called canonical or direct form 1. It is suitable for fixed-point systems due to know delay states.

There exist another form, called direct form 2 shown below.

\begin{equation}
 y[n]=b_{0}w[n]+b_{1}w[n-1]+b_{2}w[n-2]
\end{equation}
where
\begin{equation}
 w[n]=x[n]-a_{1}w[n-1]-a_{2}w[n-2]
\end{equation}  

This form is called direct form 2 and is suitable for floating point based systems.

Make transition to the other text by uno where higher order filters are used due to their superior frequency control...



Färre beräkingar

Direct form 1

Canonical Direct form 2

Led in på de olika formerna och varför biquad är speciellt intressanta...



\paragraph{Discrete structures}












\begin{itemize}
\item Digital signa processing (short)
\item General on filters
\item IRR and FIR
\item direct forms 1 and 2
\item Introduction, What is an EQ?
\item Shelving filters
\item Peak filters
\item Putting it together (conclusion)
\end{itemize}

\subsubsection{Delay}
\begin{itemize}
\item Your stuff goes here Philip
\end{itemize}

\subsection{Interface}
\label{Theory:Interface}
%In this chapter we discuss what part of the system will be taken care from team Interface. Include GUI, communication, UART and C code for handling the data on the softcore.

\subsection{Softcore}
\label{Theory:Softcore}
%In this chapter we discuss what part of the system will be taken care from the team Softcore. Include setting up the softcore, Coding and testing on the core, IP cores that will be used and the performance-power consumption management.

%%%%%%%%%%%%%%%%%%%%%%%%%%%%%%%%%%%%%%%%%%%%%%%%%%%%%%%%%%%%%%%%%%%%%

\newpage
\section{Design}
\label{Design}
% In this Chapter you discuss your proposed method or design you will follow to solve the above % problem (Question). You are also called to elaborate on the design decisions and back them % up with arguments. Finally any final results or deliverables are to be presented here.
\subsection{System overview}
\label{Design:SystemOverview}
%In this chapter we discuss the specific design methods, choises and iterrations done during the project.

\subsection{Audio}
\label{Design:Audio}
%In this chapter we discuss all the choices and actions taken during the construction of the Effects and the Sound processing kernel. All must derive from the above theory and backed up by schematics and theory.
\subsubsection{Equalizer}
\begin{itemize}
\item Fixed point for calculations.
\item Direct from 1 due to fixed
\item Relying on software FP for coefficient calculations
\item Interpolation to reach a comprimize between load and precision
\item 3-band but flexible
\item Internal precision 32 bits, crushed while maintaining sign
\end{itemize}
\subsubsection{Delay}

\subsection{Interface}
\label{Design:Interface}
%In this chapter we discuss the decisions and work flow in designing the GUI, the communication protocols and architecture and finally the UART and C code for handling the data on the softcore is to be explained.

\subsection{Softcore}
\label{Design:Softcore}
%In this chapter we discuss the method of setting up the softcore, how the C Coding and testing on the core is performed, what IP cores will be used and the performance-power consumption tradeoffs that are done in our design. Schematics from Leon3 regarding the Core architecture are a must!




%%%%%%%%%%%%%%%%%%%%%%%%%%%%%%%%%%%%%%%%%%%%%%%%%%%%%%%%%%%%%%%%%%%%%

\newpage
\section{Evaluation - Reflection}
\label{Evaluation-Reflection}
% In this Chapter you present your testing of the above design, including any itterations you did while debugging, again backed up with arguments. The last part is the evaluation and reflection on the above results and in the Task in total.



%%%%%%%%%%%%%%%%%%%%%%%%%%%%%%%%%%%%%%%%%%%%%%%%%%%%%%%%%%%%%%%%%%%%%

\section{Conclusion}
\label{Conclusion}




%-------------------------------------------------------------------
%END OF CHAPTERS
%-------------------------------------------------------------------


%--------------------------------------------------------
%REFERENCES
%--------------------------------------------------------

\newpage
\section*{References}

\bibliographystyle{IEEEtranS}
\bibliography{audioDoc.bib}



%--------------------------------------------------------
%APPENDIX
%--------------------------------------------------------
%\newpage
%\section*{APPENDIX}


\end{document}

